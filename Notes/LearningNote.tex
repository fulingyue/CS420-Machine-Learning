\documentclass[12pt,a4paper]{article}

% This is a template for lecture notes.

\usepackage{amssymb}
\usepackage[UTF8]{ctex}
\usepackage{amsmath}
\usepackage{amsthm}
\usepackage{geometry}
\usepackage{booktabs}
\usepackage{bm}
\usepackage{cite}
\usepackage{CJK}
\usepackage[many]{tcolorbox}
%\tcbuselibrary{listingsutf8}
%\tcbuselibrary{skins, breakable, theorems, most}
\geometry{a4paper,bottom = 3cm,left = 3cm, right = 3cm}
\CTEXoptions[today=old]
%for reference
\usepackage{hyperref}
\usepackage[capitalise]{cleveref}
\crefname{enumi}{}{}
\usepackage{titlesec}   %设置页眉页脚的宏包

\usepackage{
    color,
    clrscode,
    amssymb,
    amsmath,
    listings,
    xcolor,
    multirow,
    mathtools,
    mathrsfs,
    amsthm,
    systeme,
    amsfonts,
    chngcntr,
    subcaption,
    adjustbox,
    tikz-cd
}

\newtheoremstyle{mythm}{1.5ex plus 1ex minus .2ex}{1.5ex plus 1ex minus .2ex} 
    {}{\parindent}{\bfseries}{}{1em}{} 
\theoremstyle{mythm}
\newtheorem{theorem}{Theorem}
\newtheorem{lemma}[theorem]{Lemma}
\newtheorem{corollary}[theorem]{Corollary}
\newtheorem{fact}[theorem]{Fact}
\newtheorem{definition}[theorem]{Definition}
\newtheorem*{remark}{Remark}


\title{Linear Model}
\author{Fu Lingyue}
\date{\today}
 
 
\begin{document} 
\newpagestyle
{main}{               
     \sethead{Machine Learning}{\thepage}{\today}     %设置页眉    
    % \setfoot{}{\thepage}{}      %设置页脚,可以在页脚添加 
    % \thepage                      %显示页数    
    \headrule                % 添加页眉的下划线  
    \footrule    %添加页脚的下划线
}

\pagestyle{main}    %使用该style 


\begin{center}
    \Huge
    \textbf{CS420 Machine Learning}
     
\end{center}

\begin{center}
    \par March 2020
    \par Author: Fu Lingyue  \\Mail: fulingyue@sjtu.edu.cn
\end{center}

\tableofcontents
\newpage


\section{Prequisitions}
\subsection{ML's Goal}
Machine learning has a loss function $\mathcal{L}$, 
and we have to let it take the minimum by learning the parameters of $f_\theta$.
$$ 
\mathrm{min} \{\frac{1}{N} \Sigma_{i = 1}^N \mathcal{L} (y_i,f_\theta(x_i))\}
$$

是预测结果接近真实的标签,是机器学习总体的目标。

\subsection{Outline of Proper Nouns}
\paragraph{欠拟合、过拟合}
算法无法捕捉数据基础变化趋势时,出现欠拟合;
模型把随机误差和噪声也考虑进去时,出现过拟合。

\paragraph{正则化(Regularization)} Add a parameter($\lambda \Omega(\theta)$) penalty to prevent the model from overfitting the data.

L2 regularization(Ridge): $\Omega(\theta) = ||\theta||^2_2 = \Sigma_{m=1}^M\theta_m^2$.

L1 regularization(Lasso): $\Omega(\theta) = ||\theta||_1 = \Sigma_{m=1}^M|\theta_m|.$

\paragraph{交叉验证(Cross Validation)} The training data were randomly divided into k groups, and every time we use one group to verify our model.


\paragraph{模型泛化性(Model Generalization)} 

\paragraph{判别模型和生成模型} 判别模型关注数据中的一维(显式函数);生成模型关注数据之间的联系(隐函数).









\input{LinearModel.tex}
    
\end{document}